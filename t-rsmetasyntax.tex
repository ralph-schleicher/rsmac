\documentclass{article}
\usepackage[utf8]{inputenc}
\usepackage[T1]{fontenc}
\usepackage{rsmetasyntax}

\begin{document}
\section{Verbatim Text}
Execute the shell command ‘\code{ls -la~~\#\ that's\ hairy}’ to see all
files.

Execute the (more text before \code{\\code} to fill the line)
shell command ‘\code*{ls -la \var{dir}  \#\ that's  hairy}’
to see all files.

All characters in \meta{argument} print itself except for the escape
character \code\\ and the grouping characters \code\{ and \code\}.
Two control sequences have another meaning than usual; \code{\\\ }
prints a visible space and \code{\\\\} prints a backslash character.

\textbackslash{} at the beginning of a line.  The grouping
characters \{ and \} have to nest properly.

\code\\ at the beginning of a line.  The grouping
characters \code\{ and \code\} have to nest properly. (code)


\section{Symbol Names}

$|$\symb{\code}$|$ \\
$|$\symb{\code*}$|$ \\
$|$\symb{foo_bar}$|$ \\
$|$\symb{*foo-bar*}$|$ \\
$|$\symb{+foo-bar+}$|$ \\
$|$\symb{struct dirent}$|$


\section{Meta-Syntax}

An environment for typesetting meta-syntax rules.  Can be used for
documenting \TeX{} commands and \LaTeX{} environments.

\begin{metasyntax}
\start\meta{rule}\=\meta{non-terminal}\\=\meta{expression}
\meta{expression}\=\meta{terminal}\|\meta{non-terminal}\|\meta{sequence}\|\meta{alternative}
\&\|\meta{group}\|\meta{expression}\,\meta{kleene-operator}
\meta{terminal}\=\meta{character}\+
\meta{non-terminal}\=\\meta\{\meta{name}\}
\meta{sequence}\=\meta{expression}\(\\,\meta{expression}\)\+
\meta{alternative}\=\meta{expression}\(\\|\meta{expression}\)\+
\meta{group}\=\\(\meta{expression}\\)\|\\group\{\meta{expression}\}
\meta{kleene-operator}\=\\*\|\\+\|\\?
\meta{character}\=a\dots z\|A\dots Z\|0\dots 9
\&\|!\|"\|#\|$\|%\|&\|'\|(\|)\|*\|+\|,\|-\|.\|/\|:\|;\|<\|=\|>\|?\|@\|[\|]\|^\|_\|`\||\|~
\&\|\meta{visible-space}\|\meta{backslash}\|\meta{left-brace}\|\meta{right-brace}\|\meta{dots}
\meta{visible-space}\=\\\textvisiblespace
\meta{backslash}\=\\\\
\meta{left-brace}\=\\\{
\meta{right-brace}\=\\\}
\meta{dots}\=\\dots
\end{metasyntax}
%%$

Rules are placed on a line by itself.  Use \code{\\\&} at the beginning
of a line to continue a long rule.  Rules are usually wrapped before
an alternative ‘\code{\\|}’ meta-symbol.

Add \code{\\start} at the beginning of a line to mark the start rule
of a grammar.
\end{document}
