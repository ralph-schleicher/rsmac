\documentclass{article}
\usepackage{blindtext}
\usepackage{rsdisplay}

\begin{document}
\noindent
\verb|{display}| : \blindtext[1]
\begin{display}
First line.
Second line.
The rest of it is just there to add more contents to the displayed text.
\end{display}
\verb|{display}*| : \blindtext[1]
\begin{display}*
First line.
Second line.
The rest of it is just there to add more contents to the displayed text.
\end{display}
\verb|{display}[big]| : \blindtext[1]
\begin{display}[big]
First line.
Second line.
The rest of it is just there to add more contents to the displayed text.
\end{display}
\verb|{display}[medium]| : \blindtext[1]
\begin{display}[medium]
First line.
Second line.
The rest of it is just there to add more contents to the displayed text.
\end{display}
\verb|{display}[small]| : \blindtext[1]
\begin{display}[small]
First line.
Second line.
The rest of it is just there to add more contents to the displayed text.
\end{display}
\verb|{display}[none]| : \blindtext[1]
\begin{display}[none]
First line.
Second line.
The rest of it is just there to add more contents to the displayed text.
\end{display}
\blindtext[1]
\end{document}
