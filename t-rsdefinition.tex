\documentclass{article}
\usepackage[utf8]{inputenc}
\usepackage[T1]{fontenc}
\usepackage{blindtext}
\usepackage{rsdefinition}

\makeatletter
\parskip\smallskipamount
\makeatother

\begin{document}
\blindtext[1]
\begin{definition}
\defun{user-name}
Function without an argument.
\end{definition}
\blindtext[1]
\begin{definition}
\defun{forward-line} &optional \var{n}
Move \var{n} lines forward (backward if \var{n} is negative).

Returns the count of lines left to move.  If moving forward, that is
\var{n} minus number of lines moved; if backward, \var{n} plus number
of lines moved.
\end{definition}
\blindtext[1]
\begin{definition}
\defun{read} \var{stream} &optional (\var{eof-error-p} \symb{t}) \var{eof-value}

An environment for documenting commands and application programming
interfaces.  Inside a definition environment, use the \code{\\deffn},
\code{\\defvr}, and \code{\\deftp} commands to describe functions,
variables, data types, and other such artifacts.  The first line is
the signature line of the definition.  The actual documentation
follows in the body of the definition environment.  Here are two
examples:

\begin{verbatim}
\begin{definition}
\defun{forward-line} &optional \var{n}
Move \var{n} lines forward (backward if \var{n} is negative).
\end{definition}
\end{verbatim}

\begin{definition}
\defun{forward-line} &optional \var{n}
Move \var{n} lines forward (backward if \var{n} is negative).
\end{definition}

\begin{verbatim}
\begin{definition}
\defun[double]{sin} (double \var{x})
\defun[double]{cos} (double \var{x})
Return the sine or cosine of \var{x} respectively.
\end{definition}
\end{verbatim}

\begin{definition}
\defun*[double]{sin} (double \var{x})
\defun*[double]{cos} (double \var{x})
Return the sine or cosine of \var{x} respectively.
\end{definition}

\begin{definition}
\defun[double]{sin} (double \var{x})
\defun[double]{cos} (double \var{x}, double \var{x}, double \var{x}, double \var{x}, double \var{x}, double \var{x}, double \var{x}, double \var{x})
Return the sine or cosine of \var{x} respectively.
\end{definition}
\end{definition}

\blindtext[1]

No blank lines around this one.
\begin{definition}
\defun*[void *]{fubar}{\var{first-arg}
  &key
  \var{second-arg} \var{third-arg}
  (\var{fourth-arg} \symb{fourth-value})
  (\var{fifth-arg} \symb{fifth-value})
  (\var{sixth-arg} \symb{sixth-value})}
\defun[void *]{fubar}{%
  \var{foo} \var{bar} \var{baz} \var{hack} \var{tem}
  \var{foo} \var{bar} \var{baz} \var{hack} \var{tem}
  \var{foo} \var{bar} \var{baz} \var{hack} \var{tem}
  \var{foo} \var{bar} \var{baz} \var{hack} \var{tem}
  \var{foo} \var{bar} \var{baz} \var{hack} \var{tem}
  \var{foo} \var{bar} \var{baz} \var{hack} \var{tem}
  \var{foo} \var{bar} \var{baz} \var{hack} \var{tem}
  \var{foo} \var{bar} \var{baz} \var{hack} \var{tem}
  \var{foo} \var{bar} \var{baz} \var{hack} \var{tem}
  \var{foo} \var{bar} \var{baz} \var{hack} \var{tem}}
Two functions with many arguments.
\end{definition}
\blindtext[1]

\begin{definition}
\defun[double] {round_up} (double \var{number}, double \var{scale})
\defun[double] {round_down} (double \var{number}, double \var{scale})
\defun[double] {round_zero} (double \var{number}, double \var{scale})
\defun[double] {round_inf} (double \var{number}, double \var{scale})
Round number to a multiple of \var{scale}.
\begin{itemize}
\item
First argument \var{number} is a number.
\item
Second argument \var{scale} is the rounding scale factor.
Value has to be a positive number.
\end{itemize}
Return value is the rounded number.
\begin{description}
\item[\symb{round_up}]
%\symbitem{round_up}
Round towards plus infinity.
\symbitem{round_down}
Round towards minus infinity.
\symbitem{round_zero}
Round towards zero (away from infinity).
\symbitem{round_inf}
Round away from zero (towards infinity).
\end{description}
\end{definition}

\begin{definition}
\deftex{\code}\meta{argument}
\deftex{\code*}\meta{argument}
Print \meta{argument} as literal text in a typewriter type font.
The special characters
\begin{display}[small]
\textbackslash\quad\{\quad\}\quad\#\quad\$\quad\%\quad\&\quad\textasciicircum\quad\_\quad\textasciitilde
\end{display}
\noindent
have to be escaped.  Two control sequences have another meaning than
usual; \code{\\\ } prints a visible space and \code{\\\\} prints a
backslash character.  The star variant allows line breaks.

As a rule of thumb, single word commands, options, and symbol names
should be printed without quotes.  Examples on the other hand, should
be printed with single quotes.
\end{definition}

\begin{definition}
\defenv{metasyntax}
An environment to describe the syntax of a language, for example, a
computer programming language, an application programming interface, a
file or document format, or a communication protocol.

Regular text is printed as in the \code{\\code} command and such texts
denote terminals.  Non-terminals should be formatted with the
\code{\\meta} command.  A set of meta-symbol commands is available
inside the body of the \code{metasyntax} environment denoting syntax
contructs.
\end{definition}
\end{document}
